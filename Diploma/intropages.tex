% !TeX spellcheck = sl_SI

% stran 1 med uvodnimi listi
\thispagestyle{empty} 
\pagenumbering{Roman}
\begin{center}
{\large 
UNIVERZA V LJUBLJANI\\
FAKULTETA ZA MATEMATIKO IN FIZIKO\\
ODDELEK ZA FIZIKO
}

Smer študija: matematično-fizikalna

\vspace{5cm}
{\Large Klemen Kenda}\\

\vspace{2cm}
\textbf{\LARGE Uporaba metod strojnega učenja pri analizi}
\vspace{0.1cm} 

\textbf{\LARGE podatkov, zajetih z detektorjem ATLAS}

\vspace{1cm}
{ \Large Diplomsko delo }\\

\vspace{6cm} 
{\Large Mentor: prof. dr. Tomaž Podobnik} \\
{\Large Somentor: dr. Andrej Gorišek}

\vfill
{\Large Ljubljana, 2016}
\end{center}

\newpage

%stran 2 med uvodnimi listi (prazna)
\ \thispagestyle{empty}
\newpage

% stran 3 med uvodnimi listi
\thispagestyle{empty}

\vspace{1cm}
\begin{center} 
{\Large \textbf{IZJAVA O AVTORSTVU IN OBJAVI}}
\end{center}

\begin{center} 
{\Large elektronske oblike diplomskega dela}
\end{center}

\vspace{1cm}
Podpisani Klemen Kenda izjavljam:

\begin{itemize}
	\item da je diplomsko delo z naslovom \textit{Uporaba metod strojnega učenja pri analizi podatkov, zajetih z detektorjem ATLAS} rezultat mojega samostojnega dela pod mentorstvom prof. dr. Tomaža Podobnika in somentorstvom dr. Andreja Goriška,
	\item da je tiskani izvod dela identičen elektronskemu izvodu in 
	\item da Fakulteti za matematiko in fiziko Univerze v Ljubljani dovoljujem objavo elektronske oblike svojega dela na spletnih straneh Repozitorija Univerze v Ljubljani.
\end{itemize}

\vspace{1cm}
\hspace{9cm} Podpis avtorja:

\vspace{2cm}

Ljubljana, 5. septembra 2016 
\newpage

%stran 4 med uvodnimi listi (prazna)
\ \thispagestyle{empty}
\newpage

% stran 5 med uvodnimi listi
\thispagestyle{empty}

$\;$ 

\vspace{5cm}
\hfill {\Large \em Očetu.}
\thispagestyle{empty}

\vfill
{\LARGE \textbf{Zahvala}}
\vspace{0.5cm}

Za pomoč pri pripravi diplomskega dela se zahvaljujem mentorju prof. dr. Tomažu Podobniku in somentorju dr. Andreju Gorišku. Za dodaten pregled poglavij, povezanih s strojnim učenjem, in obilico koristnih namigov se zahvaljujem Roku Močniku in prof. dr. Dunji Mladenić. Za pozoren pregled celotnega besedila se zahvaljujem dr. Ajdi Flašker. Predvsem doc. dr. Mihi Škarabotu, prof. dr. Dušanu Petroviču in mag. Ani Pribaković Borštnik, ki so svoje spodbudne besede prenesli v konkretna dejanja, pa gre zahvala, da so se stvari na dolgotrajni poti do diplome na koncu le premaknile z mrtve točke. Takšne zahvale si zagotovo ne zasluži kolega Grega Milčinski.



\newpage
\setlength{\parindent}{0.7cm}
\setlength{\parskip}{0.1cm}


% -----------------------------------------------------------------------------
% POVZETEK
% -----------------------------------------------------------------------------
\chapter*{Povzetek}
\addcontentsline{toc}{chapter}{Povzetek}
\noindent Diplomsko delo obravnava problem ločevanja signala in ozadja v eksperimentu ATLAS pri razpadu $H \rightarrow \tau^+\tau^-$. Reševanje problema temelji na skupku algoritmov, ki jih v računalništvu imenujemo tudi \textit{strojno učenje}. Algoritmi so uspešni pri iskanju področij faznega prostora meritev, ki ustrezajo signalu.

Zaradi podobnosti z drugimi razpadi, je razpad $H \rightarrow \tau^+\tau^-$ še posebej težaven za klasifikacijo. Diplomsko delo se naslanja na podatke in nekatere izsledke javnega izziva \textit{HiggsML}, ki so ga z namenom iskanja najboljših algoritmov za klasifikacijo razpadov $H \rightarrow \tau^+\tau^-$ organizirali sodelavci kolaboracije ATLAS.

V diplomskem delu je predstavljen kratek opis problema. Izbrane relevantne metode strojnega učenja, ki so jih identificirali organizatorji izziva, so predstavljene in preizkušene na podatkih simulatorja detektorja ATLAS. Poleg preizkusa omenjenih metod sem v diplomskem delu razvil lasten algoritem, ki temelji na metodi podpornih vektorjev.

Uspešnost razvitega algoritma presega nekatere osnovne implementacije metode pospešenega gradientnega spusta na odločitvenih drevesih, ki jo trenutno uporabljajo pri klasifikaciji dogodkov na detektorju ATLAS, vseeno pa glede na izbrano metriko krepko zaostaja za zmagovalnimi rešitvami.


\vspace{1.3cm}

\noindent {\large \bf Ključne besede:}

\vspace{0.1cm}
\noindent strojno učenje, nadzorovano učenje, klasfikacijske metode, logistična regresija, pospešeni gradientni spust, odločitvena drevesa, metoda podpornih vektorjev, SVM, Higgsov bozon

\chapter*{Abstract}
\addcontentsline{toc}{chapter}{Abstract}
\vspace{1.3cm}
\noindent Diplomma is focused on the problem of separating signal of decay $H \rightarrow \tau^+\tau^-$  from the background in the ATLAS experiment. Solution of the problem is based on algorithms that belong to the field of \textit{machine learning} in  computer science. Algorithms are successful and efficient at searching the phase space of ATLAS measurements that correspond to the signal.

Due to similarities with other decays, the signal from decay $H \rightarrow \tau^+\tau^-$ is particularly difficult to classify. Diplomma work is based on the data and some of the achievements from the public \textit{HiggsML} challenge, which was organised by the ATLAS staff in order to find better algorithms to classify decays $H \rightarrow \tau^+\tau^-$.

A short description of the problem is provided in the diplomma. Selected relevant machine learning methods that have been suggested by the challenge organisers, were described and tested on the data from the ATLAS detector simulator. I have also developed my own algorithm for solving the problem of separating signal from the background, that is based on the \textit{support vector machine} method.

Accuracy of the developed algorithm exceeds some of the basic applications of the gradient boosted trees, which are currenty used for event classification in the ATLAS detector. However, in the selected metric it does not match the winning methods.
\vspace{1.3cm}

\noindent {\large \bf Keywords:}
\vspace{0.1cm}

\noindent machine learning, supervised learning, classification methods, logistic regression, gradient boosting, decision trees, support vector machines, SVM, Higgs boson
\vspace{1.3cm}

\noindent {\large \bf PACS:}
\vspace{0.1cm} 

\noindent 89.20.Ff, 14.80.Bn

\newpage

\renewcommand\thepage{} 
\tableofcontents 
\renewcommand\thepage{\arabic{page}}
\thispagestyle{empty}


\newpage

\mbox{}
\thispagestyle{empty}
\newpage

\setcounter{page}{1}
\pagenumbering{arabic}


%stran 8 med uvodnimi listi (prazna)
\thispagestyle{empty}
\newpage


